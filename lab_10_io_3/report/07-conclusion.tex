\section*{Заключение}

Таким образом, при буферизованном вводе-выводе все данные и при чтении, и при записи пишутся сначала в буфер, а только после этого в файл. При отсутствии буферизации данные сразу пишутся в/читаются из файла.

В программе с \textit{fopen()}, \textit{fprintf()} (с буферизацией) информация записывается из буфера в файл при вызове \textit{fclose()}. Поэтому размер файла до вызова \textit{fclose()} по данным из \textit{stat()} равен 0.

В программе с \textit{open()}, \textit{write()} (без буферизации) информация записывается в файл при вызове \textit{write()}. Поэтому размер файла до вызова \textit{fclose()} по данным из \textit{stat()} равен 13 — столько символов в итоге останется в файле.
\newpage