\documentclass[a4paper,14pt]{extarticle}

\usepackage{cmap} % Улучшенный поиск русских слов в полученном pdf-файле
\usepackage[T2A]{fontenc} % Поддержка русских букв
\usepackage[utf8]{inputenc} % Кодировка utf8
\usepackage[english,russian]{babel} % Языки: русский, английский

\usepackage{enumitem} % Настройка оформления списков
\setlist{nolistsep} 
\setenumerate[1]{label={\arabic*)}}

\usepackage{enumitem}


\usepackage[14pt]{extsizes} % Задание 14-размера шрифта

\usepackage{caption} % Подпись картинок и таблиц
\captionsetup{labelsep=endash} % Разделитель между номером и текстом краткое тире и пробел
\captionsetup[figure]{name={Рисунок}} % Изменяет имя для всех фигур на "Рисунок"

\usepackage{amsmath} % Что-то связанное с математикой
\usepackage{amssymb}

\usepackage[left=3cm,right=1.5cm,top=2cm,bottom=2cm]{geometry} % Задание геометрии листа

\usepackage{titlesec} % Оформление заголовков
\titleformat{\section}[block]
{\bfseries\large}
{\thesection}
{1em}
{}

\titleformat{\subsection}[hang]
{\bfseries\normalsize}
{\thesubsection}
{1em}{}

\titleformat{\subsubsection}[hang]
{\bfseries\normalsize}
{\thesubsubsection}
{1em}{}

\usepackage{setspace}
\onehalfspacing % Полуторный интервал

\frenchspacing
\usepackage{indentfirst} % Красная строка

\usepackage{listings} % Оформление листингов
\usepackage{xcolor} % Добавление цветов 

\usepackage{graphicx} % Вставка рисунков

\newcommand{\img}[4] {
	\begin{figure}[ht!]
		\center{
			\includegraphics[height=#1]{../data/img/#2}
			\caption{#3}
			\label{img:#4}
		}
	\end{figure}
}

\usepackage[justification=centering]{caption} % Настройка подписей float объектов

\usepackage[unicode,pdftex]{hyperref} % Ссылки в pdf
\hypersetup{hidelinks}

\usepackage{csvsimple}
\usepackage{float}

%\usepackage[shortcuts]{extdash}

%\counterwithin{figure}{section}
%\counterwithin{table}{section}
%\numberwithin{equation}{section}

\usepackage{makecell}
\renewcommand\theadfont{}

\renewcommand{\labelitemi}{---}
\usepackage{multirow}

% Оформление листингов
\usepackage{listings}
\usepackage{xcolor} % Добавление цветов 

\lstdefinestyle{mystyle}{ % Опеределение стиля 
	language=C++,
	backgroundcolor=\color{white},
	basicstyle=\footnotesize\ttfamily,
	keywordstyle=\color{blue},
	stringstyle=\color{red},
	commentstyle=\color{gray},
	numbers=left,
	numberstyle=\tiny,
	stepnumber=1,
	numbersep=5pt,
	frame=single,
	tabsize=4,
	captionpos=t,
	breaklines=true,
	breakatwhitespace=true,
	xleftmargin=10pt
}
\lstset{extendedchars=true, texcl=true}
\lstset{style=mystyle}

\newcommand{\mylisting}[4] {
	\noindent
	{
		\captionsetup{justification=raggedright,singlelinecheck=off}
		\begin{lstinputlisting}[
			caption={#3},
			label={lst:#4},
			linerange={#2}
			]{../../src/#1}
		\end{lstinputlisting}
}}